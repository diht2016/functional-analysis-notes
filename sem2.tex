\documentclass[main]{subfiles}

\begin{document}

\begin{exercise}
  Пусть $1 \le p \le \infty$, $\frac{1}{p} + \frac{1}{q} = 1$, тогда
  \begin{enumerate}
    \item $\left(\Real^n_p\right)^* \cong \Real^n_q$,
      $f(x) = "(x, y)" = \sum_{k=1}^n x_k y_k$
    \item $(l_p)^* \cong l_q$
      $f(x) = \sum_{k = 1}^\infty x_k \overline{y_k}$.
      (бесконечности соответствует $c_0$ "--- пространство
      последовательностей, сходящихся к нулю, с супремумной метрикой;
      $(c_0)^* \cong l_1$)
    \item $\left( L_p[a, b] \right)^* \cong L_q[a, b]$
    \item $\left( C[a,b] \right)^* \cong \widetilde{BV}[a,b]$
      (функции ограниченной вариации), $f(x) = \int_a^b x(t) dy(t)$ "---
      интеграл Стильтьеса
  \end{enumerate}
\end{exercise}

\begin{theorem}[Хан-Банах]
  Пусть $E$ "--- ЛНП, $M \subset E$ "--- линейное многообразие,
  $f$ "--- линейный ограниченный функционал на $M$. Тогда
  его можно непрерывно продолжить на всё $E$, сохранив норму.
\end{theorem}
\begin{proof}
  Предположим, что $E$ "--- сепарабельное вещественное пространство.
  Продолжим функционал на "<новую размерность">: если
  $M \ne E$, то $\exists x_0 \notin M$.
  Обозначим $M_1 = M \oplus [ x_0 ]$, тогда $y = x + \alpha x_0$
  и $f_1(y) = f_1(x) + \alpha f(x_0)$. Осталось выбрать
  $a = f(x_0)$ так, что $||f_1|| = ||f||$.
  
  Итак, мы хотим достичь неравенства $|f_1(y)| \le ||f|| \cdot ||y||$,
  т. е. $|f(x) + \alpha a| \le ||f|| \cdot ||x + \alpha x_0||$,
  где $x \in M$ и $\alpha \in \Real$. Иначе это будет эквивалентно
  неравенству $|f(\frac{x}{\alpha}) + a| \le ||f|| \cdot ||\frac{x}{\alpha} + x_0||$.
  Обозначим $z = \frac{x}\\alpha \in M$, т. е. мы хотим
  достичь неравенства $|f(z) + a| \le ||f|| \cdot ||z + x_0||$ для произвольного
  $z \in M$, что эквивалентно $-||f|| \cdot ||z + x_0|| \le f(z) + a \le ||f|| ||z + x_0||$
  или $-||f|| \cdot ||z + x_0|| - f(z) \le a \le ||f|| \cdot ||z + x_0|| - f(z)$.
  
  Покажем, что для произвольных $z_1$ и $z_2$
  $-||f|| \cdot ||z_1 + x_0|| - f(z_1) \le ||f|| \cdot ||z_2 + x_0|| - f(z_2)$,
  или $f(z_2) - f(z_1) \le ||f|| (||z_1 + x_0|| + ||z_2 + x_0||)$.
  Действительно, $|f(z_2) - f(z_1)| = |f(z_2 - z_1)| \le ||f|| \cdot ||z_2 - z_1|| =
  ||f|| \cdot ||(z_2 + x_0) - (z_1 + x_0)|| \le ||f|| (||z_2 + x_0|| + ||-z_1 - x_0||)$.

  Пусть теперь $X = \{ x_k \}_{k = 1}^\infty$ "--- плотное в $E$ множество,
  обозначим $M_0 = M$ и $M_{n + 1} = M_n + [x_{n+1}]$ для $n \in \Natural$.
  Поэтапно мы можем продолжить $f$ на каждый $M_n$, а потому и на
  $M_\infty = \bigcup M_n$ как $f_\infty$, при том $||f_\infty|| = ||f||$.
  Конечно, $M_\infty \supset X$ плотно в $E$, а потому мы можем продолжить
  $f_\infty$ на $E$ с сохранением нормы по (TODO: ссылка на теорему).
\end{proof}

\begin{corollary}
  Пусть $E$ "--- ЛНП.
  \begin{enumerate}
    \item $M \subset E$ "--- линейное многообразие, $E \ne M$,
      $x_0 \notin \overline{M}$. Тогда $\exists f \in E^*$ такой,
      что $M \subset \Ker f $, $f(x_0) = 1$ и $||f|| = \frac{1}{\rho(x_0, M)}$.
    \item Для произвольного $x \ne 0$ существует $f \in E^*$ такой, что
      $f(x) = ||x||$.
    \item Если $\Forall{f \in E^*} f(x) = f(y)$, то $x = y$.
    \item $\Forall{x \in E} ||x|| = \sup_{||f|| = 1} |f(x)|$.
  \end{enumerate}
\end{corollary}
\begin{proof}
  Доказательство первого пункта остаётся в качестве упражнения.

  \begin{description}
    \item[$(1) \to (2)$]
      Положим $M = \{ 0 \}$ и $x_0 = \frac{x}{||x||}$. Тогда
      найдётся $f$ такой, что $f(\frac{x}{||x||}) = 1 \To
      f(x) = ||x||$ и $||f|| = \frac{1}{\rho(\frac{x}{||x||}, {0})} = 1$.
    \item[$(2) \to (3)$]
      Пусть $z = x - y$ и $\Forall{f \in E^*} f(z) = 0$.
      Если $z \ne 0$, то $||z|| \ne 0$, и тогда по следствию $(2)$
      существует $f_0 \in E^*$ такой, что $f_0(z) = ||z|| \ne 0$,
      что противоречит условию. Значит, $z = 0$ и $x = y$.
    \item[$(2) \to (4)$]
      Пусть $f_0$ "--- функционал, удовлетворяющий условиям пункта $(2)$,
      тогда
      \[ ||x|| = f_0(x) \le \sup_{||f|| = 1} |f(x)| \le
	\sup_{||f|| = 1} ||f|| \cdot ||x|| =
      \sup_{||f|| = 1} 1 \cdot ||x|| = ||x||. \]
  \end{description}
\end{proof}

\begin{remark}
  Второй пункт следствия утверждает существование опорной гиперплоскости для шара в любой точке.

  Пусть $E$ "--- ЛНП над $\Real$.
  Для любой точки $x_0 \in S(0, 1)$ существует гиперплоскость,
  проходящая через $x_0$ и такая, что шар $\overline{B}(0, 1)$
  лежит от неё по одну сторону.

  Применим следствие $(2)$ к точке $x_0$, выберем $f \in E^*$,
  $||f|| = 1$ и $f(x_0) = ||x_0| = 1$. Тогда
  $\Forall{x \in \overline{B}(0, 1)} f(x) \le |f(x)| \le
  ||f|| \cdot ||x|| = 1 \cdot 1 \le 1$, т. е. нам подходит
  гиперплоскость $f(x) = 1$.
\end{remark}

\begin{exercise}
  $E$ "--- ЛНП, $\overline{B}(0, 1)$, $x_0 \notin \overline{B}(0, 1)$.
  Доказать, что существует гиперплоскость, разделяющая $\overline{B}$
  и $x_0$.
\end{exercise}

\begin{remark}
  \begin{itemize}
    \item Является ли продолжение в теореме Хана-Банаха
      единственным? Нет, см. задание.
    \item Можно ли обобщить результат на более широкий класс
      функционалов? Да, см. Колмогорова, Фомина;
      опорная гиперплоскость для произвольного выпуклого множества.
    \item Можно ли обобщить результат на линейные операторы?
  \end{itemize}
\end{remark}

\subsection{Изометрическое вложение $E$ в $E^{**}$}
Пусть $E$ "--- ЛНП, рассмотрим отображение $\pi : E \to E^{**}$,
переводящее $x \in E$ в функционал $F_x : f \mapsto f(x)$.
Тогда по следствию (4)
\[
  ||F_x|| = \sup_{||f|| = 1} |F_x(f)| =
  \sup_{||f|| = 1} |f(x)| = ||x||.
\]

Заметим: не всегда $\pi E = E^{**}$; если же это равенство выполнено,
то $E$ называется \emph{рефлексивным}. Например, для $p, q > 1$,
$\frac1p + \frac1q = 1$, $(l_p)^* = l_q$.

\begin{exercise}
  Если ЛНП $E$ рефлексивно, то $E$ "--- банахово.
\end{exercise}

\end{document}

\documentclass[main]{subfiles}

\begin{document}
\section{Спектр и резольвента} % 7
%Воспоминание: в линейной алгебре
%мы изучали собственные значения линейных операторов,
%т. е. такие числа \( \lambda \),
%что для некоторого ненулевого \( x \)
%\[ Ax = \lambda x; \]
%очевидно, это эквивалентно выполнению равенства
%\[ \det (A - \lambda I) = 0. \]
%
%% связь с ОТА, её доказательство в ТФКП

Далее в этом параграфе,
если не указано обратное,
\( E \) "--- банахово пространство над \( \Complex \),
\( A \in \Linears{E} \) и \( \lambda \in \Complex \).
Также введём обозначение \( A_\lambda \coloneqq A - \lambda I \).

Также в этом параграфе мы будем активно
пользоваться результатами,
полученными в курсе
теории функций комплексного переменного.
В нём мы рассматривали
исключительно функции
вида \( f : \Complex \to \Complex \);
однако, как можно догадаться по названию,
главную роль тут играет область определения функции.
Большую часть результатов можно перенести
на функции с областью значений из достаточно
широкого класса~"--- банаховых алгебр.

\begin{definition}
  \emph{Банаховой алгеброй}
  называет комплексное банахово пространство \( E \),
  снабжённое также операцией умножения
  \( \cdot : E \times E \to E \),
  которая является 
  ассоциативной, билинейной и
  обладает нейтральным элементом \( e \),
  при этом \( ||e|| = 1 \)
  и для произвольных \( x, y \in E \)
  \( ||x \cdot y|| \le ||x|| \cdot ||y|| \).
\end{definition}

Значительная часть определений и теорем в ТФКП
переносится тривиальным образом на случай
банаховых алгебр, заменой в нужных местах
умножения комплексных чисел на умножение в алгебре
и модуля на норму.

Примером банаховой алгебры является
\( \Linears{E} \), где в качестве операции
умножения выступает композиция операторов;
как раз с операторами мы и будем работать в этом параграфе.

%Самое важная для приложений — существование
%ОНБ из с. в. для эрмитового оператора.
%Например, для использования метода Фурье решения ДУ.
%Функция Грина для перехода в интегральные, потом Гильберт-Шмидт,
%
%\begin{theorem*}[Гильберт--Шмидт]
%  Пусть \( H \) "--- сепарабельное гильбертово пространство,
%  \( A \in \Linears{H} \) "--- компактный самосопряжённый оператор.
%  Тогда в \( H \) найдётся ортонормированный базис,
%  состоящий из состоящий из собственных векторов \( A \).
%\end{theorem*}

\begin{definition}
  Точка \( \lambda \in \Complex \) называется \emph{регулярной}
  точкой оператора \( A \), если
  \( \exists (A - \lambda I)^{-1} \in \mathcal{L}(E) \);
  множество всех регулярных точек оператора \( A \)
  называют \emph{резольвентным множеством} и
  обозначают как \( \rho(A) \).
  Совокупность всех \( \lambda \in \Complex \)
  не являющихся регулярными
  называют \emph{спектром} оператора \( A \) и
  обозначают с помощью \( \sigma(A) \).
\end{definition}

\begin{remark}
  Если \( \dim E < \infty \),
  то \( \sigma(A) \) есть в точности
  множество собственных значений оператора \( A \).
\end{remark}

\begin{definition}
  Определим \emph{точечный}, \emph{непрерывный}
  и \emph{остаточный} спектры оператора \( A \) как,
  соответственно, множества
  \begin{align}
    \sigma_P(A) &= \{ \lambda \in \Complex \mid
      \Ker A_\lambda \ne \{ 0 \}
    \}, \\
    \sigma_C(A) &= \{ \lambda \in \Complex \mid
      \Ker A_\lambda = \{ 0 \}, \Img A_\lambda \ne E,
      \Cl{\Img A_\lambda} = E
    \}, \\
    \sigma_R(A) &= \{ \lambda \in \Complex \mid
      \Ker A_\lambda = \{ 0 \}, \Cl{\Img A_\lambda} \ne E
    \}. \\
  \end{align}
\end{definition}

\begin{remark}
  В силу теоремы Банаха об обратном операторе,
  \( \sigma(A) = \sigma_P(A) \sqcup \sigma_C(A) \sqcup \sigma_R(A) \).
\end{remark}

\begin{definition}
  \emph{Спектральным радиусом} оператора \( A \) называется
  \[
    r(A) \coloneqq \sup_{\lambda \in \sigma(A)} |\lambda|.
  \]
\end{definition}

\begin{theorem}\label{thm:spectrum}
  Пусть \( E \) "--- комплексное банахово пространство,
  \( A \in \mathcal{L}(E) \).
  Тогда \( \sigma(A) \) "--- непустое замкнутое множество
  и
  \[
    r(A) =
    \lim_{n \to \infty} \sqrt[n]{||A^n||}.
  \]
\end{theorem}

Для удобства разобьём доказательство
этой теоремы на несколько утверждений.

%$\mathcal{L}(E)$ "--- банахова алгебра.
%>= 1972 В. М. Тихомиров

\setcounter{proposition}{-1}
\begin{proposition} % 0
  \( \rho(A) \) "--- открытое множество.
\end{proposition}
\begin{proof}
  Нужно показать, что для произвольного
  \( \lambda_0 \in \rho(A) \) некоторая её
  окрестность будет целиком состоять
  из регулярных точек.
  Заметим, что
  \[
    A_\lambda = A - \lambda I =
    A - \lambda_0 I + \lambda_0 I - \lambda I =
    A_{\lambda_0} + (\lambda_0 - \lambda) I;
  \]
  из теоремы о возмущённом операторе мы получаем,
  что \( A_\lambda \) обратим при
  \( |\lambda_0 - \lambda| < ||A_{\lambda_0}^{-1}||^{-1}|| \).
  Таким образом,
  \( B(\lambda_0, ||A_{\lambda_0}^{-1}||^{-1}||) \subset \rho(A) \).
\end{proof}

%\begin{proposition}
%  $|\lambda| > ||A|| \To \lambda \in \rho(A)$.
%\end{proposition}
%\begin{proof}
%  $A - \lambda I = - \lambda (I - \frac{1}{\lambda} A)$,
%  и тогда, т. к. $||-\frac{1}{\lambda} A|| < 1$,
%  по теореме существует обратный $-\frac1\lambda \sum \frac1{\lambda^n} A^n$.
%\end{proof}
%

\begin{definition}
  \emph{Резольвентой} оператора \( A \) называется
  отображение \( \lambda \to A_\lambda^{-1} \),
  заданное на \( \rho(A) \); обозначение: \( R_\lambda \).
\end{definition}

\begin{proposition} % 1
  \( R_\lambda \) непрерывна на \( r(A) \).
\end{proposition}
\begin{proof}
  Вспомним, как выглядит ряд Неймана
  для обращения возмущённого оператора:
  \[
    (A + \Delta)^{-1} = \sum_{n=0}^\infty (-1)^n (A^{-1} \Delta)^n A^{-1}.
  \]
  С помощью формулы суммы геометрической прогрессии
  легко получить оценку на качество аппроксимации первым членом:
  \[
    ||(A + \Delta)^{-1} - A^{-1}|| \le \frac{||A^{-1}||^2 ||\Delta||}
    {1 - ||A^{-1}|| \cdot ||\Delta||}.
  \]

  Возьмём в качестве исходного оператора \( A_{\lambda_0} \),
  и добавим к нему \( \Delta = (\lambda_0 - \lambda) I \).
  При \( \lambda \) достаточно близких к \( \lambda_0 \)
  ряд Неймана сходится и мы получаем неравенство
  \[
    ||R_\lambda - R_{\lambda_0}|| \le
    \frac{||R_{\lambda_0}||^2 |\lambda_0 - \lambda|}
    {1 - ||R_{\lambda_0}|| \cdot |\lambda_0 - \lambda|};
  \]
  очевидно, правая часть стремится к нулю при
  \( \lambda \to \lambda_0 \), т. е.
  \( R_\lambda \to R_{\lambda_0} \).
\end{proof}

\begin{proposition}[тождество Гильберта] % 2
  Пусть \( \lambda_0, \lambda \in \rho(A) \), тогда
  \[
    R_\lambda - R_{\lambda_0} = (\lambda - \lambda_0)
    R_\lambda R_{\lambda_0}.
  \]
\end{proposition}
\begin{proof}
  Интуицию нужно искать в приведении дробей к общему знаменателю.
  Если вместо операторов рассматривать числа, то всё очевидно:
  \[
    \frac1{A - \lambda \cdot 1} - \frac{1}{A - \lambda_0 \cdot 1} =
    \frac{(A - \lambda_0) - (A - \lambda)}{(A - \lambda)(A - \lambda_0)} =
    \frac{\lambda - \lambda_0}{(A - \lambda)(A - \lambda_0)}.
  \]
  Для случая операторов нужно вспомнить, что, по определению,
  \( R_\lambda A_\lambda = I = A_\lambda R_\lambda \),
  а потому
  \[ R_\lambda - R_{\lambda_0} = 
    R_\lambda A_{\lambda_0} R_{\lambda_0} -
    R_\lambda A_\lambda R_{\lambda 0} =
    R_\lambda (A_{\lambda_0} - A_\lambda) R_{\lambda_0} =
    R_\lambda (\lambda I - \lambda_0 I) R_{\lambda_0} =
    (\lambda - \lambda_0) R_\lambda R_{\lambda_0}.
  \]
\end{proof}

\begin{proposition} % 3
  \( R_\lambda \) дифференцируема на \( \rho(A) \).
\end{proposition}
\begin{proof}
  С использованием предыдущих утверждений доказательство тривиально:
  \[
    \lim_{\lambda \to \lambda_0}
    \frac{R_\lambda - R_{\lambda_0}}{\lambda - \lambda_0} =
    \lim_{\lambda \to \lambda_0}
    \frac{(\lambda - \lambda_0) R_\lambda R_{\lambda_0}}{\lambda - \lambda_0} =
    \lim_{\lambda \to \lambda_0} R_\lambda R_{\lambda_0} =
    R_{\lambda_0} \lim_{\lambda \to \lambda_0} R_\lambda = R_{\lambda_0}^2.
    \qedhere
  \]
\end{proof}

Ряд Неймана
\[
  -\frac{1}{\lambda} \sum_{n=0}^\infty \frac{1}{\lambda^n} A^n.
\]
Ряд Лорана.

Отметим, что
\( R_\lambda = (A - \lambda I)^{-1} = -\frac1\lambda
(I - \frac1\lambda A)^{-1} \),
и по теореме~\ref{thm:inverse-neumann}
при \( \lambda > ||A|| \) \( R_\lambda \)
определена и равна
\[ -\frac1\lambda \sum_{n=1}^\infty \frac1{\lambda^n} A^n. \]
На самом деле, радиус сходимости этого
ряда может быть уточнён.

\begin{proposition} % 4
  Радиус сходимости ряда 
  \[
    -\frac1\lambda \sum_{n=1}^\infty \frac1{\lambda^n} A^n
  \]
  равен спектральному радиусу \( r(A) \)
\end{proposition}
\begin{proof}
  Сначала покажем, что при \( |\lambda| > r(A) \)
  ряд сходится.
  Напрямую из определения \( r(A) \)
  следует, что
  \( \{ \lambda \mid |\lambda| > r(A) \} \subset \rho(A) \);
  итак, функция комплексного переменного определена
  и дифференцируема на данном кольце
  (в смысле комплексного анализа).
  Значит, она раскладывается на нём в ряд Лорана;
  из соображений единственности ряда Лорана
  для кольца \( \{ \lambda \mid |\lambda| > ||A|| \} \)
  следует, что ряд Лорана для
  \( \{ \lambda \mid |\lambda| > r(A) \} \)
  является рядом из условия
  теоремы.
 
  С другой стороны, при \( |\lambda| < r(A) \)
  ряд будет расходится.
  Действительно, если это не так для некоторого \( \lambda_0 \)
  удовлетворяющего условию, то
  по первой теореме Абеля
  ряд будет сходится на
  множестве
  \( \{ \lambda \mid |\lambda| > |\lambda_0| \} \).
  При этом, его сумма всё ещё будет
  являться \( A_\lambda^{-1} \in \Linears{E} \),
  т.~е. при \( |\lambda| > |\lambda_0| \)
  \( \lambda \) "--- регулярная точка.
  Значит, \( \Forall{\lambda \in \sigma(A)}
  |\lambda| \le |\lambda_0| < r(A) \),
  что противоречит определению супремума.
\end{proof}

\begin{proposition} % 5
  \( \sigma(A) \ne \emptyset \).
\end{proposition}
\begin{proof}
  Предположим противное.
  Тогда \( \rho(A) = \Complex \)
  и \( R_\lambda \) "--- целая функция
  (определённая и дифференцируемая на \( \Complex \)).
  Ряд Лорана сходится для \( \lambda > 0 \) и
  при \( \lambda \to \infty \)
  \[
    ||R_\lambda|| =
    ||-\frac{1}{\lambda} \sum_{n=0}^\infty \frac{1}{\lambda^n} A^n|| \le
    \frac{1}{|\lambda|} \sum_{n=0}^\infty \frac{1}{|\lambda|^n} ||A||^n|| =
    \frac{1}{\lambda} \cdot \frac{1}{1 - \frac{||A||}{|\lambda|}} \to 0.
  \]
  Следовательно, \( R_\lambda \) ограничена,
  и тогда по теореме Лиувилля
  она константна.
  Из той же оценки \( ||R_\lambda|| \) следует,
  что в таком случае \( R_\lambda = 0 \).
  Это, конечно, невозможно в случае нетривиального
  пространства \( E \).
\end{proof}

\begin{proposition} % 6
  \( \lambda \in \sigma(A) \To \lambda^n \in \sigma(A^n) \).
\end{proposition}

\begin{proof}
  Предположим противное: \( \lambda^n \in \rho(A^n) \), т. е.
  \( \Exists {(A^n - \lambda^n I)}^{-1} \in \Linears{E} \).
  Тогда умножив равенство
  \[
    (A^n - \lambda^n I) = (A - \lambda I)
    (A^{n -1} + \lambda A^{n-2} + \dots + \lambda^{n-1} I)
  \]
  справа на \( {(A^n - \lambda^n I)}^{-1} \) мы покажем,
  что \( (A^{n -1} + \dots + \lambda^{n-1} I) {(A^n - \lambda^n I)}^{-1} \)
  "--- правый обратный оператор для \( (A - \lambda I) \).
  Выполняя симметричные действия мы получаем и левый обратный оператор.
  Значит, \( A_\lambda \) обратим \( \lambda \notin \sigma(A) \),
  что противоречит условию.
\end{proof}

\begin{exercise}
  \( \sigma(A^n) = {(\sigma(A))}^n \).
\end{exercise}

\begin{proposition}
  $ r(A) = \lim \sqrt[n]{||A^n||} $.
\end{proposition}
\begin{proof}
  Мы уже показали, что спектральный радиус равен радиусу сходимости
  ряда Лорана для \( R_\lambda \) на бесконечности;
  с другой стороны, радиусом сходимости этого ряда определяется
  формулой Коши-Адамара, а потому
  \[
    r(A) = \limsup_{n \to \infty} \sqrt[n]{||A^n||};
  \]
  теперь достаточно показать, что у этой последовательности есть предел.

  Из прошлого утверждения следует,
  что \( {\left( r(A) \right)}^n \le r(A^n) \);
  кроме того, \( r(A^n) \le ||A^n|| \),
  и потому \( r(A) \le \sqrt[n]{||A^n||} \).
  Если обозначить \( \sqrt[n]{||A^n||} \) как \( \alpha_n \),
  то мы получили неравенства
  \[
    \limsup_{n \to \infty} \alpha_n \le \alpha_n
  \]
  для каждого \( n \). Перейдём к нижнему пределу:
  \[
    \limsup_{n \to \infty} \alpha_n \le \liminf_{n \to \infty} \alpha_n;
  \]
  очевидно, нижний предел не может быть больше верхнего,
  значит, они совпадают. В таком случае,
  последовательность попросту сходится,
  что и требовалось.
\end{proof}

Из объединения приведённых выше утверждений
напрямую следует теорема~\ref{thm:spectrum}.

\begin{exercise}
  Оператор Вольтерра в \( C[0, 1] \) определяется так:
  \[ (Af)(x) = \int_{0}^x f(t) dt. \]
  Докажите, что \( r(A) = 0 \), \( \sigma(A) = \{ 0 \} \)
  и \( 0 \in \sigma_R(A) \).
\end{exercise}
 
\end{document}

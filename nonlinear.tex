\documentclass[main]{subfiles}

\begin{document}

\section{Элементы нелинейного анализа}%15

В вещественном анализе мы говорили,
что функция дифференцируема в точке \( x_0 \), если
\[
  f(x) = f(x_0) + A (x - x_0) + o(x - x_0).
\]
С другой стороны, имелось определение производной
как предела \( \frac{f(x) - f(x_0)}{x - x_0} \),
и они были тесно связаны.
Аналогичная ситуация наблюдалась в многомерном
и комплексном анализе, но в бесконечномерном
случае эта связь теряется.

\begin{definition}
  Пусть \( E_1 \), \( E_2 \) "--- нормированные вещественные пространства,
  \( D \) "--- область в \( E_1 \), \( F: D \to E_2 \).
  \( F \) \emph{дифференцируема (по Фреше)} в точке \( x_0 \in D \),
  если
  \[
    \Exists{A} \in \Linears{E_1, E_2}
    \lim_{h \to 0} \frac{||F(x_0 + h) - F(x_0) - Ah||}{||h||} = 0.
  \]
\end{definition}

\begin{definition}
  В тех же условиях определим в точке \( x_0 \) для \( h \in E_1 \)
  дифференциал Гато
  \[
    DF(x_0; h) = \frac{d}{dt} F(x_0 + t h) \bigr|_{t=0}.
  \]
  Если \( DF(x_0; \cdot) \in \Linears{E_1, E_2} \),
  то \( F \) \emph{дифференцируема по Гато} в точке \( x_0 \).
\end{definition}

\begin{exercise}
  Из дифференцируемости по Фреше следует дифференцируемость по
  Гато, но не наоборот.
\end{exercise}

\begin{proposition}~
  \begin{enumerate}
    \item \( F(x) \equiv const \To \Forall{x_0} F'(x_0) = 0 \)
    \item \( F \in \Linears{E_1, E_2} \To
      \Forall{x_0} F'(x_0) = F(x_0) \)
    \item \( (\alpha_1 F_1 + \alpha_2 F_2)' =
      \alpha_1 F_1' + \alpha_2 F_2' \)
    \item \( (H \circ G \circ F)' = H' \circ G' \circ F' \)
  \end{enumerate}
\end{proposition}

\begin{theorem}
  Пусть \( E_1 \), \( E_2 \) "--- вещественные НП,
  \( D \subset E_1 \) "--- выпуклая область
  и \( F : D \to E_2 \) дифференцируема на \( D \)
  (по Фреше).
  Тогда
  \[
    ||F(x_1) - F(x_0)|| \le \sup_{\xi \in [x_0, x_1]} ||F'(\xi)|| \cdot ||x_1 - x_0||.
  \]
\end{theorem}
\begin{proof}
  Для \( x \in [0, 1] \) определим \( x_t = x_0 + (1 - t) (x_1 - x_0) \) и
  для произвольного \( f \in E_2^* \) рассмотрим композицию
  \( \phi = f \circ F \circ x_\cdot \)
  (\( \phi : [0, 1] \to \Real \)).
  \( f \) дифференцируем как любой линейный функционал,
  \( F \) дифференцируема по условию
  и \( x_t' = x_1 - x_0 \).
  Значит, дифференцируема и \( \phi \),
  и по теореме Лагранжа о среднем
  \( |\phi(1) - \phi(0)| = \phi'(s) \)
  для некоторого \( s \in [0, 1] \).
  По второму следствию из теоремы Хана-Банаха
  найдётся \( f \in E^2 \) такой, что ...
  и тогда
  \( ||F x_1 - F x_0|| = |f(Fx_1 - F x_0)| =
  |\phi(1) - \phi(0)| = \phi'(s) \le
  \sup |F'(\xi)| \cdot ||x_1 - x_0|| \).
\end{proof}

\begin{definition}
  Отображение (не обязательно линейное)
  \( F : E_1 \to E_2 \)
  называется компактным, если
  \( F \) непрерывно и для любого ограниченного
  множества \( A \subset E_1 \)
  \( F(A) \) "--- предкомпактно в \( E_2 \).
\end{definition}

\begin{theorem}[Шаудера]
  Пусть \( E(\Real) \) "--- банахово пространство,
  \( D \subset E \) "--- замкнутое выпуклое множество
  и отображение \( F : D \to D \) компактно.
  Тогда у него существует неподвижная точка.
\end{theorem}

\begin{example}
  Доказать, что краевая задача
  \[
    \begin{cases}
      y'' + \lambda \sin y = f(x), & x \in (0, 1) \\
      y(0) = y(1) = 0
    \end{cases}
  \]
  имеет решение для любых \( \lambda \in \Real \)
  и \( f \in C[0, 1] \).
\end{example}

\end{document}
